\documentclass[a4paper, 12pt]{report}
\usepackage[portuguese]{babel}
\usepackage{chemist}
\usepackage{chemformula}
\usepackage{booktabs}
% Pacote para aceitar acentuação
\usepackage[utf8]{inputenc}
\usepackage{float}
% geometry, pacote para configurar as margens. 
% padrão ABNT => lmargin=3cm,tmargin=3cm,rmargin=2cm,bmargin=2cm
\usepackage[lmargin=3cm,tmargin=3cm,rmargin=2cm,bmargin=2cm]{geometry}

% graphicx = usar imagem (\includegraphics), comment = texto em formato de comentário, enumitem = enumeração, multirow e multicol = uso de tabelas, indentf = identar primeira linha
\usepackage{graphicx,xcolor,comment,enumitem,multirow,multicol,indentfirst}
\setlength{\parindent}{1.25cm} % não identa a primeira linha

% Pacotes para matemática
\usepackage{amsmath, amssymb, amsthm, amsfonts,  mathtools, amstext, nccmath}
\everymath{\displaystyle}

\usepackage[utf8]{inputenc}
\usepackage{hyperref}

%
% Título e autor do documento
%
\title{Química Analítica Quantitativa\\
	\large{\textbf{\\Volumetria de Neutralização \\
	Titulação de Ácidos Fracos com Base Forte (NaOH)}}
}

\author{Rogério Ribeiro Macêdo}
\date{\today}

\usepackage[font=scriptsize,labelfont=bf]{caption}
\usepackage{subcaption}
\usepackage{siunitx}

\usepackage{fancyhdr}
\fancyhf{} % limpa os cabecalhos e rodapés
\fancyhead[C]{Titulação de Ácidos Fracos com Base Forte (NaOH)} % define o cabeçalho personalizado
\renewcommand{\footrulewidth}{0.5pt}
\fancyfoot[C]{\thepage}
\pagestyle{fancy} % sem definir esse comando, o cabeçalho personalizado não é exibido

\begin{document}

\maketitle
\tableofcontents
\newpage
\chapter{Titulação}
O presente trabalho foi realizado objetivando aplicar a sequência de cálculos necessários à titulação de ácido fraco (titulado) com uma base forte, no caso estamos usando o hidróxido de sódio (titulante) como base forte. A partir dos cálculos o resultado final é a produção da curva de titulação. Lembrando que, para tal, por se tratar de um ácido fraco faz-se necessário a utilização da constante de dissociação do ácido (Ka). \\

A curva de titulação será construída relacionando o volume adicionado de NaOH com o valor do pH resultante da titulaçao. 

\section{Etapas da titulação}
Antes de iniciar vale lembrar que o processo de titulação envolve basicamente 4 etapas e em cada uma delas o cálculo do pH é realizado considerando o contexto da titulação:

\begin{description}
	\item[\underline{Etapa 1}]: antes de iniciar a titulação \hfil \\ Neste ponto a solução contém apenas ácido acético. Portanto, o valor do pH é determinado pela dissociação do ácido.
	\item[\underline{Etapa 2}]: antes de atingir o ponto de equivalência \hfil \\ Ainda há ácido para reagir, assim, o pH é determinado pelo sistema tampão.
	\item[\underline{Etapa 3}]: no ponto de equivalência \hfil \\ O ponto de equivalência é o ponto onde todo o ácido reagiu com a base. O volume da base para isso é aquele que será determinado pelo volume de equivalência. O pH é determinado pela hidrólise do sal formado.
	\item[\underline{Etapa 4}]: depois do ponto de equivalência \hfil \\ Há excesso de base. O pH é determinado através desse excesso. A hidrólise do sal contribui pouco nesse ponto, pois o excesso de base reprime esta reação.
\end{description}

\section{Lista de Ácidos}

Foi realizado um levantamento de 10 ácidos (monopróticos) para que pudéssemos avaliar o comportamento da curva de titulação destes. Abaixo apresenta-se a lista de ácidos:

\begin{table}[H]
	\begin{center}
		\begin{tabular}{lp{5cm}lp{5cm}lp{5cm}lp{5cm}}\toprule
			& \textbf{Ácido} & \textbf{Fórmula} & \textbf{Ka} & \textbf{pKa} \\ \midrule
			& Ácido acético & \chemform{CH_{3}COOH} & $1,75 \times 10^{-5}$ & 4,756 \\
			& Ácido benzoico & \chemform{C_{6}H_{5}CO_{2}H} & $6,25 \times 10^{-5}$ & 4,204 \\
			& Ácido ciânico & \chemform{HCNO} & $3,50 \times 10^{-4}$ & 3,460 \\
			& Ácido fórmico & \chemform{CH_{2}O_{2}} & $1,80 \times 10^{-4}$ & 3,750 \\
			& Ácido hidrazoico & \chemform{HN_{3}}. & $2,50 \times 10^{-5}$ & 4,602 \\
			& Ácido hidrociânico & \chemform{HCN} & $6,20 \times 10^{-10}$ & 9,207 \\
			& Ácido fluorídrico & \chemform{HF} & $6,30 \times 10^{-4}$ & 3,200 \\
			& Ácido iodoacético & \chemform{CH_{2}ICO_{2}H} & $ 6,60 \times 10^{-4}$ & 3,180 \\
			& Ácido nitroso & \chemform{HNO_{2}} & $5,60 \times 10^{-4}$ & 3,252 \\
			& Ácido hipoiodoso & \chemform{HIO} & $3,20 \times 10^{-11}$ & 10,495 \\
			\bottomrule
		\end{tabular}
	\end{center}
\end{table}


\section{Dados para titulação}

Tratando-se de um experimento teórico, optou-se por considerar os dados de concentração e volume abaixo descritos. Tais valores serão utilizados para a lista de ácidos apresentada acima.

\begin{table}[H]
	\begin{center}
		\begin{tabular}{lp{5cm}lp{10cm}}\toprule
			& \textbf{Ácido} & \textbf{Hidróxido de sódio} \\ \midrule			
			Volume & 50,00 mL & - \\ 
			Concentração & $0,100 \: mol \cdot L^{-1}$ & $0,100 \: mol \cdot L^{-1}$ \\ 
			\bottomrule
		\end{tabular}
	\end{center}
\end{table}

\section{Volume de equivalência}
O volume de equivalência é o volume necessário de titulante que irá reagir completamente com o titulado. No caso, o volume de NaOH necessário para reagir completamente com o ácido. O cálculo é realizado usando a expressão\footnote{C: concentração; V: volume}:

\begin{equation*}
	V_{titulante} \times C_{titulante} = V_{titulado} \times C_{titulado}
\end{equation*}

Portanto, o volume necessário de NaOH que será necessário para neutralizar todo o ácido será de:
\begin{fleqn}
\begin{align*}
	& V_{titulante} \times C_{titulante} = V_{titulado} \times C_{titulado} \\
	& V_{titulante} \times 0,100 = 50,00 \times 0,100 \\
	& V_{titulante} = 50,00 \text{ mL}
\end{align*}
\end{fleqn}

No caso deste trabalho e como informado acima, a concentração do ácido e da base são as mesmas, portanto, mesmo sem realizar o cálculo já poderíamos prever que o volumen necessário de base para neutralizar completamente o ácido seria de 50,00 mL.

\section{Cálculos}
A partir deste ponto realizaremos os cálculos da titulação. Como o objetivo é construir a curva de titulação, os volumes que usaremos de NaOH serão pontuais. O conjunto de dados que será utilizado para a construção da curva pode ser encontrado no link: \href{https://github.com/rogerioribeiromacedo}{https://github.com/rogerioribeiromacedo/Chemistry}\footnote{https://github.com/rogerioribeiromacedo/Chemistry}

\subsection{No ponto inicial (sem adição de NaOH)}
O valor do Ka é considerado para a realização do cálculo, e este é calculado usando a expressão abaixo:
\begin{fleqn}
\begin{align*}
	Ka = \frac{[H_{3}O^{+}] \times [Ac^{-}]}{[HAc]}
\end{align*}
\end{fleqn}
Assim:
\begin{fleqn}
\begin{align*}
	& Ka = \frac{x \times x}{[HAc]} \longrightarrow	Ka = \frac{x^{2}}{[HAc]} \rightarrow x^{2} = Ka \times [HAc] \longrightarrow	 x = \sqrt{Ka \times [HAc]} \\ \\
	& x = [H_{3}O^{+}] = \sqrt{(1,75 \times 10^{-5}) \times 0,100} \longrightarrow	\mathbf{x = [H_{3}O^{+}] = 1,32 \times 10^{-3}}
\end{align*}
\end{fleqn}
O valor do pH:
\begin{fleqn}
\begin{align*}
	& pH = - \log [H_{3}O^{+}] \\
	& pH = - \log {1,32 \times 10^{-3}} \\
	& \textbf{pH = 2,88}
\end{align*}
\end{fleqn}

\subsection{Antes do ponto de equivalência}
O cálculo do pH antes do ponto de equivalência faz uso da \textbf{Equação de Henderson-Hasselbalch}. Esta equação faz o relacionamento do pH de uma solução tampão, com o pKa, as concentrações da forma ácida (HAc) e da base conjugada ($Ac^{-}$). Abaixo a equação a ser utilizada:
\begin{fleqn}
\begin{align*}
	pH = pKa + \frac{[Ac^{-}]}{[HAc]}
\end{align*}
\end{fleqn}
Portanto, precisamos do pKa do ácido acético:
\begin{fleqn}
\begin{align*}
	& pKa = -log [Ka] \\
	& pKa = -log (1,75 \times 10^{-5}) \\
	& \textbf{pka = 4,757}
\end{align*}
\end{fleqn}

\subsection{Após adição de 0,05 mL (0,00005 L) de NaOH}
Esse volume representa o volume de uma única gota de NaOH adicionada na solução do titulante.

\begin{fleqn}
\begin{align*}
	& [Ac^{-}] = \frac{ V_{NaOH} \times C_{NaOH} }{ V_{HAc} + V_{NaOH} } = \frac{0,00005 \times 0,1}{0,05 + 0,00005} \\ \\
	& \mathbf{[Ac^{-}] = 0,0001 \text{ \textbf{mol}} \times L^{-1}}
\end{align*}
\end{fleqn}

\begin{fleqn}
\begin{align*}
	& [HAc] = \frac{ (C_{HAc} \times V_{HAc}) - (C_{NaOH} \times V_{NaOH}) }{ (V_{HAc} + V_{NaOH}) } = \frac{ (0,1 \times 0,05) - (0,1 \times 0,00005) }{ (0,05 + 0,00005) }\\ \\
	& \mathbf{[HAc] = 0,0998 \text{ \textbf{mol}} \times L^{-1}}
\end{align*}
\end{fleqn}

\begin{fleqn}
\begin{align*}
 	& pH = pKa + \log \frac{[Ac^{-}]}{[HAc]} = 4,757 + \log \frac{0,0001}{0,0998} \\
	& \textbf{pH = 1,758}
\end{align*}
\end{fleqn}

\subsection{Após adição de 0,1 mL (0,0001 L) de NaOH}
\begin{fleqn}
	\begin{align*}
		& [Ac^{-}] = \frac{ V_{NaOH} \times C_{NaOH} }{ V_{HAc} + V_{NaOH} } = \frac{0,0001 \times 0,1}{0,05 + 0,0001} \\ \\
		& \mathbf{[Ac^{-}] = 0,0001996 \text{ \textbf{mol}} \times L^{-1}}
	\end{align*}
\end{fleqn}

\begin{fleqn}
	\begin{align*}
		& [HAc] = \frac{ (C_{HAc} \times V_{HAc}) - (C_{NaOH} \times V_{NaOH}) }{ (V_{HAc} + V_{NaOH}) } = \frac{ (0,1 \times 0,05) - (0,1 \times 0,0001) }{ (0,05 + 0,0001) }\\ \\
		& \mathbf{[HAc] = 0,0996 \text{ \textbf{mol}} \times L^{-1}}
	\end{align*}
\end{fleqn}

\begin{fleqn}
	\begin{align*}
		& pH = pKa + \log \frac{[Ac^{-}]}{[HAc]} = 4,757 + \log \frac{0,0001996}{0,0996} \\
		& \textbf{pH = 2,0589}
	\end{align*}
\end{fleqn}

\subsection{Após adição de 10,00 mL (0,0001 L) de NaOH}
\begin{fleqn}
	\begin{align*}
		& [Ac^{-}] = \frac{ V_{NaOH} \times C_{NaOH} }{ V_{HAc} + V_{NaOH} } = \frac{0,1 \times 0,01}{0,05 + 0,01} \\ \\
		& \mathbf{[Ac^{-}] = 1,67 \times 10^{-2} \text{ \textbf{mol}} \times L^{-1}}
	\end{align*}
\end{fleqn}

\begin{fleqn}
	\begin{align*}
		& [HAc] = \frac{ (C_{HAc} \times V_{HAc}) - (C_{NaOH} \times V_{NaOH}) }{ (V_{HAc} + V_{NaOH}) } = \frac{ (0,1 \times 0,05) - (0,1 \times 0,01) }{ (0,05 + 0,01) }\\ \\
		& \mathbf{[HAc] = 6,67 \times 10^{-2} \text{ \textbf{mol}} \times L^{-1}}
	\end{align*}
\end{fleqn}

\begin{fleqn}
	\begin{align*}
		& pH = pKa + \log \frac{[Ac^{-}]}{[HAc]} = 4,757 + \log \frac{1,67 \times 10^{-2}}{6,67 \times 10^{-2}} \\
		& \textbf{pH = 4,156}
	\end{align*}
\end{fleqn}

\subsection{Após adição de 25,00 mL (0,025 L) de NaOH}
Esse volume corresponde à metade do volume de equivalência, portanto, nesse ponto, o valor do pH corresponde ao valor do pKa. Assim:

\begin{fleqn}
	\begin{align*}
		 \textbf{pH = pKa = 4,757}
	\end{align*}
\end{fleqn}

\subsection{Adição de 50,00 mL (0,05 L) de NaOH - no ponto de equivalência}
A adição de 50,00 mL de base quer dizer que estamos no ponto de equivalência, onde todo o ácido é neutralizado pela base. Neste ponto, o cálculo do pH é realizado pensando-se na hidrólise do sal (NaAc), cuja equação pode ser vista abaixo:

\begin{chemeqn}
	Ac^{-}_{(aq)} + H_{2}O_{(l)} \equilibarrow HAc_{(aq)} + OH^-{} 
\end{chemeqn}

Nesse sentido, a maneira de realizar o cálculo envolverá a concentração de $OH^{-}$ presente no sistema. E outra variável que também precisaremos é o valor de Kb. Assim:

\begin{fleqn}
	\begin{align*}
		[OH^{-}] = \sqrt{Kb \times C_{NaAc}}
	\end{align*}
\end{fleqn}

Para calcular a concentração do sal, que será a mesma do acetato, é preciso pensar que este tem uma relação estequiométrica de 1:1 com a base, portanto, para saber a $[Ac^{-}]$ usaremos os dados de NaOH, dessa forma:

\begin{fleqn}
	\begin{align*}
		&[NaAc] = \frac{ (C_{NaOH} \times V_{NaOH}) }{ (V_{HAc} + V_{NaOH}) } = \frac{ (0,1 \times 0,05) }{ (0,05 + 0,05) } \longrightarrow	 \mathbf{[NaAc] = 5,0 \times 10^-2 \text{ \textbf{mol}} \times L^{-1}}
	\end{align*}
\end{fleqn}

O próximo passo é calcular o valor do Kb: 

\begin{fleqn}
	\begin{align*}
		&Kw = Ka \times Kb \longrightarrow	 (1,0 \times 10^{-14}) = (1,75 \times 10^{-5}) \times Kb \\ \\
		&Kb = \frac{1,0 \times 10^{-14}}{1,75 \times 10^{-5}} \longrightarrow	 \mathbf{Kb = 5,71 \times 10^{-10}}
	\end{align*}
\end{fleqn}

Finalizado com o cálculo da $[OH^{-}]$ e consequentemente o valor do pH no ponto de equivalência:

\begin{fleqn}
	\begin{align*}
		&[OH^{-}] = \sqrt{Kb \times C_{NaAc}} = \sqrt{(5,71 \times 10^{-10}) \times (5,0 \times 10^{-2})} \\
		&[OH^{-}] = 5,34 \times 10^{-6} \longrightarrow	 pOH = 5,27 \longrightarrow	 \textbf{pH = 8,73}
	\end{align*}
\end{fleqn}

\subsection{Após a adição de 50,10 mL (0,0510 L) de NaOH}
Neste ponto todo o ácido foi neutralizado, portanto, passamos a ter um excesso de base.

\begin{fleqn}
	\begin{align*}
		&[OH^{-}] = \frac{ (C_{NaOH} \times V_{NaOH}) - (C_{HAc} \times V_{HAc}) }{ (V_{NaOH} \times V_{HAC}) } = \frac{ (0,1 \times 0,0510) - (0,1 \times 0,05) }{ (0,0510 + 0,05) } \\ \\	
		&[OH^{-}] = 9,99 \times 10^{-4} \longrightarrow	 pOH = 3,00 \longrightarrow	 \textbf{pH = 10,00}
	\end{align*}
\end{fleqn}

\section{Curva de titulação}
A curva de titulação (Figura \ref{fig:curvasdetitulacaoacaceticoehidroxidodesodio}) é resultado de programa escrito em Python, que, utilizando os passos descritos na seção \textbf{Etapas de titulação}, produziu os dados necessários para sua construção. \\

O programa toma como referência os dados para titulação da seção 3 e calcula o pH considerando a variação de volume do NaOH de 0,00 mL até 120 mL. Na Tabela \ref{demo-table} podemos ver os primeiros e últimos valores deste cálculo:

\begin{table}[H]
	\begin{center}
		\caption{Valores de pH nos respectivos volumes de NaOH}
		\label{demo-table}
		\begin{tabular}{rp{5cm}rp{5cm}}\toprule
			& \textbf{pH} & \textbf{Volume (L)} \\ \midrule			
			0 & 2.87848000 & 0.00000000 \\ 
			1 & 1.75740000 & 0.00005000 \\ 
			2 & 2.05886000 & 0.00010000 \\
			3 & 2.23539000 & 0.00015000 \\
			4 & 2.36076000 & 0.00020000 \\
			& ... & ... \\
			2405 & 12.61556000 & 0.12025000 \\
			2406 & 12.61574000 & 0.12030000 \\
			2407 & 12.61592000 & 0.12035000 \\
			2408 & 12.61610000 & 0.12040000 \\
			2409 & 12.61628000 & 0.12045000 \\
			\bottomrule
		\end{tabular}
	\end{center}
\end{table}

\section{Gráfico}
Abaixo pode-se visualizar o gráfico resultante dos cálculos. Nele, marcado com linha pontinhada, temos os valores de:

\begin{table}[H]
	\begin{center}
		\caption{Volume equivalência e pH}
		\label{tabela_vol_eq_e_pH}\textbf{}
		\begin{tabular}{lp{5cm}lp{10cm}}\toprule
			\textbf{Volume de equivalência} & \textbf{pH} \\ \midrule
			50,00 mL & 8.728 \\ 
			\bottomrule
		\end{tabular}
	\end{center}
\end{table}

%\begin{figure}[H]
%	\centering
%	\includegraphics[width=0.99\linewidth]{curvas_de_titulacao_ac_acetico_e_hidroxido_de_sodio}
%	\caption[Curva de Titulação]{}
%	\label{fig:curvasdetitulacaoacaceticoehidroxidodesodio}
%\end{figure}


\end{document}